\section{Introduction}
\label{sec:intro}

%serverless

Serverless computing is a new paradigm that makes microservices development easy and enables cloud platforms to achieve fine-grained resource allocation.
It short, it enables brief bursts of computation without specification or configuration of server resources.
%having to worry about setting up servers, so the term \textit{serverless computing}.
Many cloud providers support data center infrastructure with serverless computing capabilities, such as Amazon AWS Lambda~\cite{AWSLambda2017}, IBM Apache OpenWhisk~\cite{ApacheOpenWhisk}, Microsoft Azure Functions~\cite{AzureFunctions}, Google Cloud Functions~\cite{GoogleCloudFunctions}, and other providers using the open source system OpenLambda~\cite{OpenLambda2016}.

%Offload

Offloading computation to network elements brings many advantages since it requires fewer servers and enables more applications per server. A recent study~\cite{powersmartnics} showed that by offloading computation to network elements, operators can reduce CAPEX and OPEX costs. For 2,200 network functions per site, it is possible to save (\$12 M). It also decreases a site's carbon footprint via improved energy-efficiency.


%Contributions
In this paper, we present eBPFlow, a serverless network packet processing system. It offloads serverless computation to a NetFPGA, which is an FPGA board for network packet processing.
% For instance, Microsoft cloud Azure provides SmartNICs based on FPGAs~\cite{AzureFPGA2018}.
%Challenges
In designing eBPFlow, we faced many challenges.
We need to support reprogrammability on the fly and zero downtime to support scheduling requests.
NetFPGA, although energy efficient, imposes limitations on computation resources. The maximum clock rate is 200~MHz and the design must take the longest computational path into account since that determines the clock rate. 
NetFPGA's synthesizer also imposes some design constraints. 
Moreover, to operate at line rate, the design needs to be optimized and zero copy packet processing is a requirement.
Furthermore, as noted in~\cite{bifulco2018survey}, another challenge is to find an expressive yet simple model to handle state operations in the data plane.

% ~\cite{Levai2018}
% "The 5G core will depend on programmable switches, which allow packet processing functionality to be reconfigured on the fly in order to deploy virtualized network functions and service chains instantaneously."

\textbf{Contributions.} To solve these challenges, eBPFlow is built around a virtual engine.
It borrows the well-defined Instruction Set Architecture (ISA) of eBPF (extended Berkeley Packet Filter), which follows a RISC model and is already available in the Linux kernel and well-known in the computer networking community, providing compatibility with existing network projects.
Moreover, to improve performance, we are the first to present a 5-stage pipelined eBPF processor.
Futhermore, to operate with zero downtime, eBPFlow utilizes a double buffer system.
We design an abstract data type DM\_FIFO that is simultaneously a Data Memory and FIFO to support zero copy. 
To handle state operations in the data plane with an expressive but yet simple model, eBPFlow utilizes key-store data storage called maps that are decoupled from the data memory and can be accessed through function calls in a well-defined API.
To support multi-tenancy, we developed a scheduler in the backend to move code in and out of the NetFPGA.


%energy
%related work
There has been little prior work leveraging hardware acceleration for serverless execution.
The only related work, to our knowledge, is E3~\cite{liu2019e3}, which provides a microservice execution platform based on a SmartNIC to improve energy-efficiency. We are the first to propose a network packet processing system for serveless computing using a NetFPGA. eBPFlow consumes only $0.55$~W/Gbps while an 1U x86 server~\cite{x86serverIBM} with 4x 40~Gbps interfaces consumes on average $1.875$~W/Gbps, 3x more. 
%In Section~\ref{sec:results}, we show that SmartNICs are less energy efficient than eBPFlow.



%roadmap

This paper is organized as follows. In \textsection\ref{sec:architecture}, we present the overall architecture of eBPFlow. In \textsection\ref{sec:design}, we provide the eBPFlow packet processor. In \textsection\ref{sec:implementation}, we describe the implementation details of eBPFlow built on top of the NetFPGA SUME platform. In \textsection\ref{sec:results}, we show the evaluation and results in a realistic environment. 
%In \textsection\ref{sec:discussion}, we provide some analysis and discuss some limitations.
In \textsection\ref{sec:relatedWork}, we describe and compare the related work. Finally, in \textsection\ref{sec:conclusion}, we present the conclusions.

% \red{SHOULD I Add anything else?}

