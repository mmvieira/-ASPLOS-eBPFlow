\section{Introduction}
\label{sec:intro}

%serverless

Serverless computing is a new paradigm that allows microservices.
It enables short lived computation without having to worry about setting up servers, so the term \textit{serverless computing}.

Many companies today support data center infrastructure with serverless computing capabilities, such as Amazon AWS Lambda~\cite{AWSLambda2017},IBM Apache OpenWhisk~\cite{ApacheOpenWhisk}, Microsoft Azure Functions~\cite{AzureFunctions}, Google CloudFunctions~\cite{GoogleCloudFunctions}, and even academia with open source OpenLambda~\cite{OpenLambda2016}.

%Offload

Offloading computation to network elements brings many advantages since it requires fewer servers and enables more applications per server. A study~\cite{powersmartnics} showed that by offloading computation to network elements, it can reduces CAPEX and OPEX costs. For 2,200 network functions per site, it claims to save (\$12 M). It also minimize carbon footprint and improve energy-efficient.

Microsoft cloud Azure provides SmartNICs based on FPGAs~\cite{AzureFPGA2018}.

\marcos{Contributions}
In this paper, we present eBPFlow, a serverless network packet processing system. It offloads serverless computation to the NetFPGA.


%Challenges
In designing eBPFlow, we faced many challenges.
We need to support reprogrammability on the fly and zero downtime to support scheduling requests.
NetFPGA, although energy efficient, imposes limitation on computation resources. The maximum clock rate is 200~MHz and the design should take the longest computational path into account since that determines the clock rate. 
Moreover, to operate at line rate, the design needs to be optimized and zero copy is a requirement.
Furthermore, as stated in~\cite{bifulco2018survey}, another challenge is to find an expressive yet simple model to handle state operations in the data plane.

~\cite{Levai2018}
"The 5G core will depend on programmable switches, which allow packet processing functionality to be reconfigured on the fly in order to deploy virtualized network functions and service chains instantaneously."


%energy

E3~\cite{liu2019e3} provides a microservice execution platform based on SmartNIc to improve energy-efficient. We are the first to propose a network packet processing system for serveless computing using NetFPGA. eBPFlow consumes only $5.5$~W per interface. In Section~\ref{sec:results}, we show that SmartNICs is less energy efficient than eBPFlow.

